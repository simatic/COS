% Created 2017-10-28 sam. 16:23
\documentclass[11pt]{article}
\usepackage[utf8]{inputenc}
\usepackage[T1]{fontenc}
\usepackage{fixltx2e}
\usepackage{graphicx}
\usepackage{longtable}
\usepackage{float}
\usepackage{wrapfig}
\usepackage{rotating}
\usepackage[normalem]{ulem}
\usepackage{amsmath}
\usepackage{textcomp}
\usepackage{marvosym}
\usepackage{wasysym}
\usepackage{amssymb}
\usepackage{hyperref}
\tolerance=1000
\author{Michel SIMATIC}
\date{28/10/2017}
\title{Manuel Utilisateur de l'outil C.O.S. (Collecte d'Opinions lors de Soutenances)}
\hypersetup{
  pdfkeywords={},
  pdfsubject={},
  pdfcreator={Emacs 24.5.1 (Org mode 8.2.10)}}
\begin{document}

\maketitle
\tableofcontents


\section{Introduction}
\label{sec-1}
Lorsqu'un enseignant fait passer une soutenance à certains étudiants
d'une classe, souvent, il demande aux autres étudiants de la classe
d'assister à cette soutenance pour que les étudiants qui ne
soutiennent pas :
\begin{enumerate}
\item profitent de la soutenance pour apprendre de nouvelles choses,
\item profitent des remarques qu'il fera aux étudiants en train de
soutenir,
\item fassent des remarques à leurs collègues qui soutiennent. Ainsi, ces
derniers auront un volument plus important de remarques/conseils et
pourront encore plus s'améliorer.
\end{enumerate}
Hélas, très souvent, dans la pratique, les étudiants qui ne soutiennent
pas n'écoutent pas la soutenance, car ils font autre chose.

\emph{COS} (Collecte d'Opinions lors de Soutenances) est un outil destiné
à favoriser l'attention des étudiants lors des soutenances de leur
collègues. Ainsi, les objectifs évoqués précédemment ont plus de
chances d'être atteints.

\emph{COS} a été testé sous Linux, MacOS et Windows. Il est compatible avec
LibreOffice/OpenOffice Calc et Excel.

\emph{COS} s'inscrit dans la procédure suivante qui sera détaillée dans la
suite de ce document:
\begin{itemize}
\item Phase 0 (configuration initiale) : l'enseignant configure \emph{COS} en
indiquant :
\begin{itemize}
\item sur quel système d'exploitation il travaille (Linux, MacOS, Windows),
\item quel tableur il envisage d'utiliser (\emph{Excel} ou \emph{LibreOffice
Calc}). NB: \emph{COS} n'impose pas d'utiliser un tableur, mais
l'utilisation d'un tableur facilite grandement la saisie de
certaines informations ;
\item sous quel nom il veut apparaître dans les documents de synthèse qui
seront fournis aux étudiants,
\end{itemize}
\item Phase 1 (pré-soutenance) :
\begin{itemize}
\item l'enseignant spécifie les données concernant une soutenance :
\begin{itemize}
\item Le bonus qui sera accordé à la note de soutenance des étudiants
qui ont fait la même évaluation de certains critères que le
l'enseignant ;
\item Type des critères d'évaluation (par exemple, "Fond", "Forme", etc.),
\item Liste des critères d'évaluation (par exemple, "Résultats et
recul", "Dynamisme", etc.),
\item Liste des titres des soutenances,
\item Liste des étudiants qui assisteront aux soutenances,
\end{itemize}
\item l'enseignant exécute \emph{COS} pour générer :
\begin{itemize}
\item un fichier avec des fiches d'évaluation nominatives qui seront
remplies par les étudiants,
\item un fichier avec des fiches d'évaluation génériques,
\item un canevas fichier qui lui servira à entrer ses notes et
commentaires pour les différentes soutenances.
\end{itemize}
\item l'enseignant prépare la soutenance :
\begin{itemize}
\item il recopie le fichier des fiches d'évaluation nominatives dans
un éditeur de texte collaboratif (par exemple, \href{https://framapad.org/}{framapad}) ;
\item il imprime autant de fiches d'évaluation génériques qu'il y aura
d'étudiants sans accès à l'éditeur de texte collaboratif pendant
les soutenances ;
\item il enrichit le canevas de fichier pour que la saisie de notes
et de commentaires lui soit plus facile.
\end{itemize}
\end{itemize}
\item Phase Soutenance. Pendant la soutenance :
\begin{itemize}
\item l'enseignant explique comment seront exploitées les informations
fournies par les étudiants. En particulier, il évoque le bonus
dont profiteront les étudiants qui "joueront le jeu" de remplir
ces informations ;
\item les étudiants utilisent l'éditeur de texte collaboratif ou bien
remplisse une fiche d'évaluation générique ;
\item l'enseignant saisit des notes et des commentaires.
\end{itemize}
\item Phase 2 (post-soutenance) :
\begin{itemize}
\item l'enseignant récupère le contenu du fichier qui a été rempli avec
l'éditeur de texte collaboratif ;
\item l'enseignant retranscrit les notes prises manuellement sur les
fiches d'information génériques dans ce fichier.
\item l'enseignant finit de saisir ses notes et commentaires ;
\item l'enseignant exécute \emph{COS} pour générer :
\begin{itemize}
\item Une synthèse des comentaires faits par les étudiants durant la
soutenance,
\item Un fichier donnant le détail des notes de chaqsue étudiant (note
de soutenance attribuée par l'enseignant, bonus pour bonne
évaluation de soutenance de collègues, note finale).
\end{itemize}
\end{itemize}
\end{itemize}

Dans la suite de ce document, nous commençons par présenter
l'installation de \emph{COS}, avant de détailler chacune des phases
mentionnées dans cette introduction. La dernière section explique
comment changer les noms de fichier et les valeurs utilisés par
défaut.
\section{Installation de \emph{COS}}
\label{sec-2}
\subsection{Généralités}
\label{sec-2-1}
\begin{itemize}
\item Téléchargez l'archive ZIP \verb~COS-master.zip~ de \emph{COS} sur \href{https://github.com/simatic/COS}{github}
  (L'accès à l'archive se fait en cliquant sur le bouton vert \verb~Clone   or download~ en haut à droite, puis en cliquant sur \verb~Download ZIP~).
\item Dézipper cette archive sur votre ordinateur à l'endroit qui vous
arrange.
\item \emph{COS} est un programme écrit en \emph{Python 3}. Les sous-sections
suivantes détaillent l'installation de \emph{Python 3} selon votre
système d'exploitation.
\end{itemize}
\subsection{Linux}
\label{sec-2-2}
Vous n'avez rien de spécial à faire (\emph{Python 3} est installé par
défaut sur Linux).
\subsection{MacOS}
\label{sec-2-3}
Ouvrez un terminal de commande.

Tapez la commande \verb~python3~

Si vous avez une fenêtre dont le contenu ressemble au contenu suivant
(notez les \verb~>>>~ en dernière ligne) :
\begin{verbatim}
$ python3
Python 3.5.2 (default, Nov 17 2016, 17:05:23) 
[GCC 5.4.0 20160609] on linux
Type "help", "copyright", "credits" or "license" for more information.
>>>
\end{verbatim}
alors \emph{Python 3} est déjà installé : vous n'avez rien à faire (hormis
fermer votre terminal).

Sinon, téléchargez \emph{Python 3} sur le \href{https://www.python.org/downloads/mac-osx/}{site officiel}. Puis, installez-le.
\subsection{Windows}
\label{sec-2-4}
\begin{itemize}
\item Appuyez sur la touche \verb~Windows~, puis sur la touche \verb~R~ (sans
appuyer sur la touche \verb~Majuscule~ : une fenêtre \verb~Exécuter~ apparaît.
\item Dans cette fenêtre, dans le champ \verb~Ouvrir:~, tapez \verb~cmd~. Puis,
cliquez sur \verb~OK~ : uène fenêtre \verb~C:\WINDOWS\system32\cmd.exe~
apparaît.
\item Dans cette fenêtre, tapez \verb~python~, puis appuyez sur la touche
\verb~Entrée~. 3 cas sont possibles :
\end{itemize}
\subsubsection{1er cas}
\label{sec-2-4-1}
Vous obtenez l'affichage :
\begin{verbatim}
Python 3.6.3 (v3.6.3:2c5fed8, Oct  3 2017, 17:26:49) [MSC v.1900 32 bit (Intel)] on win32
Type "help", "copyright", "credits" or "license" for more information.
>>>
\end{verbatim}

Vu qu'il y a écrit \verb~Python 3.~ au début de la première ligne, cela
signifie que \emph{Python 3} est déjà installé sur votre machine : vous
n'avez rien à faire (hormis fermer cette fenêtre).
\subsubsection{2e cas}
\label{sec-2-4-2}
Vous obtenez l'affichage :
\begin{verbatim}
'python' n’est pas reconnu en tant que commande interne
ou externe, un programme exécutable ou un fichier de commandes.
\end{verbatim}

Cela signifie que \emph{Python 3} n'est pas installé sur votre machine :
\begin{itemize}
\item Téléchargez \emph{Python 3} sur le \href{https://www.python.org/downloads/}{site officiel}
\item Lancez l'installation sur votre ordinateur : une fenêtre \verb~Python 3   (32-bit) Setup~ apparaît.
\item Dans cette fenêtre, cochez : \verb~Install laucher for all users   (recommended)~ et \verb~Add Python 3.6 to PATH~. Puis, cliquez sur
"Install Now".
\item Windows vous demande si vous autorisez cette application à apporter
des modifications à votre ordinateur. Répondez que "Oui".
\item Au bout d'un moment, une fenêtre affiche un message "Setup was
successful". Cliquez sur "Close"
\end{itemize}
\subsubsection{3e cas}
\label{sec-2-4-3}
Vous obtenez l'affichage :
\begin{verbatim}
Python 2.7.3 (default, Apr 10 2012, 23:24:47) [MSC v.1500 64 bit (AMD64)] on win32
Type "help", "copyright", "credits" or "license" for more information.
>>>
\end{verbatim}

Vu qu'il y a écrit \verb~Python 2.~ au début de la première ligne, cela
signifie que c'est \emph{Python 2} qui est installé sur votre machine et
non \emph{Python 3}. Il faut installer \emph{Python 3}, mais sans déranger
l'installation de \emph{Python 2}. Pour ce faire :

\begin{itemize}
\item Téléchargez \emph{Python 3} sur le \href{https://www.python.org/downloads/}{site officiel}
\item Lancez l'installation sur votre ordinateur : une fenêtre \verb~Python 3   (32-bit) Setup~ apparaît.
\item Dans cette fenêtre, cochez : \verb~Install laucher for all users   (recommended)~. Vérifiez que \verb~Add Python 3.6 to PATH~ est
\textbf{décoché}. Puis, cliquez sur "Install Now".
\item Windows vous demande si vous autorisez cette application à apporter
des modifications à votre ordinateur. Répondez que "Oui".
\item Au bout d'un moment, une fenêtre affiche un message "Setup was
successful". Cliquez sur "Close"
\end{itemize}

\emph{Python 3} est désormais installé sur votre machine. Mais, vous devez
préciser à \emph{COS} qu'il doit utiliser ce \emph{Python 3} et non le \emph{Python
2} auquel il accéderait spontanément :
\begin{itemize}
\item Avec un explorateur de fichier, allez dans le répertoire où vous
avez dézippé \emph{COS}.
\item Clic gauche sur le fichier \verb~cos.bat~, puis clic droit pour choisir
de l'éditer avec votre éditeur de texte (par exemple, \emph{Notepad++}).
\item Ajoutez la ligne \verb~set PATH=C:\Program Files (x86)\Python36-32\~ 
après la ligne \verb~echo off~ et enregistrez votre fichier. Il doit 
donc désormais ressembler à ceci :
\end{itemize}
\begin{verbatim}
echo off
set PATH=C:\Program Files (x86)\Python36-32\
set PYTHONPATH=collectopiniondefenses
python collectopiniondefenses/main.py -%1 configuration.txt
pause
\end{verbatim}

\section{Configuration initiale de COS}
\label{sec-3}
\subsection{Configuration du système d'exploitation et du tableur que vous utilisez}
\label{sec-3-1}
Avec un explorateur de fichier, allez dans le répertoire \verb~modeles~ de
\emph{COS}. Ce répertoire contient 4 fichiers archives. Double-cliquez sur
l'archive correspondant à votre combinaison Système d'exploitation
(Linux et MacOS OU BIEN Windows) / Tableur (LibreOffice Calc et
OpenOffice Calc OU BIEN Excel), comme indiqué dans le tableau
ci-dessous.
\begin{center}
\begin{tabular}{lll}
\hline
 & LibreOffice Calc & Excel\\
\hline
Linux/MacOS & \verb~linux_MacOS_LibreOffice_Calc.zip~ & \verb~linux_MacOS_Excel.zip~\\
\hline
Windows & \verb~windows_LibreOffice_Calc.zip~ & \verb~windows_Excel.zip~\\
\hline
\end{tabular}
\end{center}

Extrayez les fichiers de cette archive de sorte que :
\begin{itemize}
\item \verb~configuration.txt~ soit extrait dans le répertoire principal de
\emph{COS};
\item \verb~Phase_1_entree/listeCriteres.csv~ soit extrait dans le répertoire
\verb~Phase_1_entree~ de \emph{COS};
\item etc.
\end{itemize}

\subsubsection{Le coin du geek (à ne lire que si vous souhaitez en savoir plus)}
\label{sec-3-1-1}
Le fichier \verb~configuration.txt~ extrait dépend du système
d'exploitation de la manière suivante :
\begin{itemize}
\item en Linux-MacOS, le champ \verb~Encodage des fichiers utilises (lus ou   generes)~ vaut \verb~utf8~ ;
\item en Windows, il vaut \verb~windows-1252~.
\end{itemize}

Tous les fichiers extraits dépendent du système d'exploitation de la
manière suivante :
\begin{itemize}
\item en Linux-MacOS, ils sont encodés en \verb~utf8~ avec des retours à la
ligne simples ;
\item en Windows, ils sont encodés en \verb~ASCII~ (\verb~windows-1252~) avec des
retours à la ligne typiques de Windows (\emph{linefeed} suivi de
\emph{Carriage return}).
\end{itemize}

Le fichier \verb~configuration.txt~ extrait dépend du tableur de la manière
suivante :
\begin{itemize}
\item pour LibreOffice/OpenOffice Calc, le champ \verb~Separateur dans les   fichiers CSV~ vaut "," (virgule, sans les guillements) ;
\item pour Excel, il vaut ";" (point-virgule, sans les guillements).
\end{itemize}

Tous les fichiers d'extension \textbf{.csv} extraits dépendent du système d'exploitation de la
manière suivante :
\begin{itemize}
\item pour LibreOffice/OpenOffice Calc, le séparateur utilisé est une ","
(virgule, sans les guillements) ;
\item pour Excel, c'est ";" (point-virgule, sans les guillements).
\end{itemize}

Remarque : si vous travaillez avec un tableur configuré pour
interpréter les nombres décimaux à l'américaine (pi s'écrit "3.14" et
non "3,14", dans \verb~configuration.txt~, il vous faut changer le champ
\verb~Separateur utilise par votre tableur pour separer la partie entiere et la partie decimale d'un nombre décimal~ en "." (point, sans les
guillemets) à la place de "," (virgule, sans les guillemets).

\subsection{Configuration de votre nom dans les documents de synthèse générés par COS}
\label{sec-3-2}
\begin{itemize}
\item Éditez le fichier \verb~configuration.txt~ en double-cliquant dessus
avec un explorateur de fichiers.
\item Modifiez le champ \verb~Nom de l'encadrant·e~ pour y indiquer le nom sous
lequel vous souhaitez apparaître dans les documents générés par
\emph{COS} (par exemple, \verb~Jeanne Dupont~).
\end{itemize}
\begin{verbatim}
# Nom de l'encadrant·e
Jeanne Dupont
\end{verbatim}
\begin{itemize}
\item Sauvegardez le fichier.
\end{itemize}
\section{Phase 1 (pré-soutenance)}
\label{sec-4}
\subsection{Configuration de COS}
\label{sec-4-1}
\subsubsection{Bonus accordé aux étudiants}
\label{sec-4-1-1}
Lors de la phase 2, quand \emph{COS} comparera les évaluations des
étudiants et les évaluations de l'enseignant, il attribuera à chaque
étudiant un bonus par évaluation d'étudiant correspondant à
l'évaluation enseignant. Même si ce bonus ne servira qu'en phase 2,
nous vous proposons de réfléchir à la valeur de ce bonus, \textbf{dès la
phase 1}, pour pouvoir l'indiquer aux étudiants lors de la phase de
soutenance.

Pour changer la valeur de ce bonus :
\begin{itemize}
\item Éditez le fichier \verb~configuration.txt~ en double-cliquant dessus
avec un explorateur de fichiers.
\item Modifiez le champ \verb~Bonus sur note etudiant·e si l'etudiant·e juge   critere comme l'encadrant~ pour y indiquer la valeur de ce bonus
(par exemple, \verb~0.1~). NB : écrivez ce nombre décimal en notation
américaine (donc, "." (point) pour séparer la partie entière de la
partie décimale).
\end{itemize}
\begin{verbatim}
# Bonus sur note etudiant·e si l'etudiant·e juge critere comme l'encadrant
# NB : Si ce bonus est decimal, utilisez un point ('.') et non une
#      une virgule (',') pour separer la partie entiere et la pa'rtie decimale.
#      Par exemple, pour un bonus egal a pi, ecrire (sans les guillemets)
#      "3.14" et pas "3,14"
0.1
\end{verbatim}
\begin{itemize}
\item Sauvegardez le fichier.
\end{itemize}
\subsubsection{Types des critères d'évaluation}
\label{sec-4-1-2}
\emph{COS} impose de catégoriser les différents critères d'évaluation. (par
exemple, "Fond", "Forme", etc.).

Pour changer la liste des types de critères d'évaluation :
\begin{itemize}
\item Éditez le fichier \verb~Phase_1_entree/listeTypesCriteres.txt~ en
double-cliquant dessus avec un explorateur de fichiers.
\item Modifiez la liste des types de critère. L'exemple définit les types
\verb~Fond~ et \verb~Forme~, la ligne \verb~# Nom de chaque type de critère~ étant
un commentaire.
\end{itemize}
\begin{verbatim}
# Nom de chaque type de critère
Fond
Forme
\end{verbatim}
\begin{itemize}
\item Sauvegardez le fichier.
\end{itemize}
\subsubsection{Critères d'évaluation}
\label{sec-4-1-3}
\emph{COS} vous permet de personnaliser les critères d'évaluation de vos
soutenances.

Pour changer la liste des critères d'évaluation :
\begin{enumerate}
\item Éditez le fichier \verb~Phase_1_entree/listeCriteres.txt~ en
double-cliquant dessus avec un explorateur de fichiers. Cela ouvre
votre tableur.
\begin{itemize}
\item NB (lié à LibreOffice/OpenOffice Calc) : Dans le cas de
\emph{LibreOffice Calc}, une fenêtre \verb~Import de texte~ s'affiche dans
un premier temps. Veillez à ce que, dans la zone \verb~Options de      séparateur~, 1) \verb~Séparé par~ soit sélectionné, 2) seul \verb~Virgule~
soit coché.
\end{itemize}
\item Modifiez la liste des critères. L'exemple définit 10 critères (5 de
\verb~Fond~ et 5 de \verb~Forme~).
\item Sauvegardez le fichier (au format \textbf{CSV}).
\end{enumerate}
\subsubsection{Titres des soutenances}
\label{sec-4-1-4}
\emph{COS} impose de lui fournir la liste des titres des soutenances qui
vont avoir lieu. Nous vous recommandons de les lui fournir dans
l'ordre de passage envisagé (cela facilite le remplissage des fiches
par les étudiants et l'enseignant).

Pour changer la liste des types de critères d'évaluation :
\begin{itemize}
\item Éditez le fichier \verb~Phase_1_entree/listeSoutenances.txt~ en
double-cliquant dessus avec un explorateur de fichiers.
\item Modifiez la liste des soutenances. L'exemple définit 3 soutenances
\verb~Eugénie Grandet~, \verb~La Touche étoile~ et \verb~... Et mon tout est un   homme~, la ligne \verb~# Nom de chaque soutenance~ étant un commentaire.
\end{itemize}
\begin{verbatim}
# Nom de chaque soutenance
Eugénie Grandet
La Touche étoile
... Et mon tout est un homme
\end{verbatim}
\begin{itemize}
\item Sauvegardez le fichier.
\end{itemize}
\subsubsection{Liste des étudiants}
\label{sec-4-1-5}
\emph{COS} impose de lui fournir la liste des étudiants qui vont soutenir
et, sur quelle soutenance, ils vont soutenir. Notez que :
\begin{itemize}
\item Plusieurs étudiants peuvent soutenir ensemble (cf., dans l'exemple,
\verb~M. AYRAUD Pierre (dit Thomas Narcejac)~ et \verb~M. BOILEAU Pierre   Louis~ qui font la même soutenance de titre \verb~... Et mon tout est un   homme~)
\item Une soutenance de \verb~Phase_1_entree/listeSoutenances.txt~ peut ne pas
se voir mentionnée à cette étape.
\end{itemize}

Pour changer la liste des étudiants :
\begin{enumerate}
\item Éditez le fichier \verb~Phase_1_entree/listeEtudiants.txt~ en
double-cliquant dessus avec un explorateur de fichiers. Cela ouvre
votre tableur.
\begin{itemize}
\item NB (lié à LibreOffice/OpenOffice Calc) : Dans le cas de
\emph{LibreOffice Calc}, une fenêtre \verb~Import de texte~ s'affiche dans
un premier temps. Veillez à ce que, dans la zone \verb~Options de      séparateur~, 1) \verb~Séparé par~ soit sélectionné, 2) seul \verb~Virgule~
soit coché.
\end{itemize}
\item Modifiez la liste des étudiants. L'exemple définit 4 étudiants.
\item Sauvegardez le fichier (au format \textbf{CSV}).
\end{enumerate}
\subsection{Exécution de la phase 1 de COS}
\label{sec-4-2}
Le lancement de la phase 1 de \emph{COS} dépend de votre système
d'exploitation.
\subsubsection{Linux/MacOS}
\label{sec-4-2-1}
Dans le répertoire de \emph{COS}, exécutez le programme
\verb~cos_phase_1.sh~. En cas d'erreur, un message vous explique le
problème détecté : à vous de le corriger. Si tout se passe bien, \emph{COS}
affiche le message :
\begin{verbatim}
cos version 1.0.0

OK, exécution de la phase 1 terminée : les fichiers...
\end{verbatim}
\subsubsection{Windows}
\label{sec-4-2-2}
Avec un explorateur de fichiers, allez dans le répertoire de
\emph{COS}. Puis, double-cliquez sur le programme \verb~cos_phase_1.bat~ : une
fenêtre s'ouvre et affiche un message d'erreur ou bien un message de
bonne exécution (cf. exemple Linux/MacOS ci-dessus). Dans les 2 cas,
appuyez sur une touche pour fermer la fenêtre.
\subsection{Préparation de la soutenance}
\label{sec-4-3}
\subsubsection{Mise à disposition sur Internet des fiches d'évaluation nominatives}
\label{sec-4-3-1}
\begin{itemize}
\item Dans un éditeur de texte collaboratif (par exemple, \href{https://framapad.org/}{framapad}), créez
un \emph{pad} public ou privé (à vous de décider, l'essentiel étant que
les étudiants puissent y accéder).
\item Dans le répertoire \verb~Phase_1_sortie~, double-cliquez sur le fichier
\verb~listeFichesEtudiants.txt~
\item Recopiez son contenu dans le \emph{pad} créé.
\end{itemize}
\subsubsection{Impression des fiches d'évaluation génériques}
\label{sec-4-3-2}
\begin{itemize}
\item Si vous savez que des étudiants n'auront pas accès à un ordinateur
(ou une tablette, si vous estimez qu'une tablette peut permettre de
modifier le \emph{pad}) pendant la soutenance, comptez le nombre
d'étudiants dans ce cas.
\item Dans le répertoire \verb~Phase_1_sortie~, imprimez le fichier
\verb~ficheGeneriqueEtudiant.txt~ en autant d'exemplaires que nécessaire.
\end{itemize}
\subsubsection{Enrichissement du fichier canevas de notes}
\label{sec-4-3-3}
\begin{itemize}
\item Dans le répertoire \verb~Phase_1_sortie~, double-cliquez sur le fichier
\verb~Phase_1_sortie/canevasNotesEncadrant.csv~ : votre tableur s'ouvre.
\item Enrichissez ce fichier, par exemple :
\begin{itemize}
\item en changeant la largeur des colonnes
\item en changeant la colonne 1 de la ligne \verb~Ligne inutilisée (par     exemple...~
\item en mettant dans les autres colonnes de cette ligne, une formule de
calcul de somme des éléments de cette colonne
\item etc.
\end{itemize}
\item Sauvegardez le tableau obtenu dans le répertoire \verb~Phase_2_entree~ au
format standard de votre tableur (\verb~.odt~ pour LibreOffice/OpenOffice
et \verb~.xlsx~ pour Excel).
\end{itemize}
\section{Phase de soutenance}
\label{sec-5}
\begin{enumerate}
\item Indiquez aux étudiants comment ils peuvent fournir les
informations. En particulier, fournissez l'adresse du \emph{pad}, voire
un crayon pour les étudiants qui rempliront un exemplaire papier.
\item Expliquez les "règles du jeu", i.e. les informations que les
étudiants doivent fournir et comment ces informations seront
exploitées :
\begin{itemize}
\item L'étudiant doit remplir la fiche correspondant à son nom.
\item Pour chaque soutenance (hormis la sienne, évidemment), l'étudiant doit :
\begin{itemize}
\item Mettre un "+" au début de la ligne d'un critère dont il pense
que l'enseignant estimera que le critère ne nécessite aucune
amélioration (voire qu'il est impeccable).
\item Mettre un commentaire au niveau du champ
\verb~Commentaire/Justification du +~.
\item S'il estime que l'enseignant considérera qu'il n'y aucun
critère impeccable pour cette soutenance, l'étudiant doit
mettre un "+" devant un critère dont il pense que l'enseignant
estimera que le critère ne nécessite qu'une (ou des)
amélioration(s) mineure(s).
\item Même principe avec un "-" correspondant à un critère
nécessitant une (ou des) amélioration(s) majeure(s).
\end{itemize}
\item Au moment du dépouillement par l'enseignant :
\begin{itemize}
\item Un "+" n'est pris en compte que si le champ
\verb~Commentaire/Justification du +~ est rempli (intelligemment).
\item Idem pour un "-"
\item Si, pour une même soutenance, l'étudiant écrit plusieurs "+" ou
plusieurs "-", aucun "+" ne sera considéré pour cette
soutenance.
\item Idem pour les "-"
\item Si, pour une soutenance, l'étudiant remplit le champ
\verb~Commentaire/Justification du +~ sans donner de "+" à un
critère, son commentaire est ignoré.
\item Idem pour \verb~Commentaire/Justification du -~
\item Chaque "+" qui correspondant à un critère jugé impeccable par
l'enseignant rapport un bonus de \emph{nombre de point bonus que
vous avez décidé précédemment} à la note finale.
\item Idem pour chaque "-"
\item Expliquez le principe de bonus si, pour une soutenance,
l'enseignant ne met aucun "+".
\item Expliquez que c'est pareil dans le cas où il ne met aucun "-".
\end{itemize}
\end{itemize}
\item Faites passer les soutenances.
\item Remplissez votre fichier de notes/commentaires
\end{enumerate}
\section{Phase 2 (post-soutenance) :}
\label{sec-6}
\subsection{Configuration de COS}
\label{sec-6-1}
\begin{itemize}
\item Complétez le \emph{pad} avec les réponses récupérées au format papier.
\item Recopiez le contenu de votre \emph{pad} dans le fichier
\verb~reponsesEtudiants.txt~ du répertoire \verb~Phase_2_entree~
\item Sauvegardez dans le répertoire \verb~Phase_2_entree~ votre fichier de
notes/commentaires au format \textbf{CSV} et sous le nom
\verb~notesEncadrant.csv~
\item NB : si vous le souhaitez, vous pouvez changer le bonus accordé aux
étudiants. En effet, c'est seulement maintenant que sa valeur va
vraiment être exploitée.
\end{itemize}
\subsection{Exécution de la phase 2 de COS}
\label{sec-6-2}
De même que pour la phase 1, la phase 2 de \emph{COS} dépend de votre
système d'exploitation :
\begin{itemize}
\item Linux/MacOS : exécutez \verb~cos_phase_2.sh~ au lieu de \verb~cos_phase_1.sh~
  précédemment.
\item Windows : exécutez \verb~cos_phase_2.bat~ au lieu de \verb~cos_phase_1.bat~
  précédemment.
\end{itemize}

En cas d'exécution correcte, vous aurez l'affichage suivant :
\begin{verbatim}
cos version 1.0.0

OK, exécution de la phase 2 terminée : les fichiers...
\end{verbatim}
\subsection{Exploitation des fichiers générés par COS (en phase 2)}
\label{sec-6-3}
Les deux fichiers générés par \emph{COS} sont disponibles dans le
répertoire \verb~Phase_2_sortie~ :
\begin{itemize}
\item \verb~syntheseCommentaires.txt~ contient la synthèse des commentaires
faits par les étudiants et vous durant la soutenance. À vous de
décider comment l'exploiter.
\item \verb~notesEtudiants.csv~ contient le calcul des différentes notes.
\begin{itemize}
\item La note de soutenance
\begin{itemize}
\item Elle est calculée en faisant la somme des notes que vous avez
attribuée à chaque critère, pour cette soutenance.
\item Si vous préférez mettre d'autres valeurs que 0, 1 ou 2, allez
dans le fichier \verb~configuration.txt~ pour changer les champs
(NB : actuellement, ces champs doivent être forcément des
entiers) :
\begin{itemize}
\item \verb~Nombre de points quand encadrant·e estime que critere revele une bonne maitrise~ (valeur actuelle : 2)
\item \verb~Nombre de points quand encadrant·e estime que critere requiert ameliorations mineures~ (valeur actuelle : 1)
\item \verb~Nombre de points quand encadrant·e estime que critere requiert ameliorations majeures~ (valeur actuelle : 0)
\end{itemize}
\end{itemize}
\item La note de bonus (qui, rappelons-le, dépend du champ \verb~Bonus sur     note etudiant·e si l'etudiant·e juge critere comme l'encadrant~
    dans le fichier \verb~configuration.txt~)
\item La note finale du module (qui est la somme de ces deux notes).
\item À vous de décider comment exploiter \verb~notesEtudiants.csv~.
\end{itemize}
\end{itemize}
\section{Changement des noms ou des valeurs utilisés par défaut}
\label{sec-7}
\subsection{Changement des noms de fichier}
\label{sec-7-1}
\subsubsection{configuration.txt}
\label{sec-7-1-1}
Si vous souhaitez que le fichier \verb~configuration.txt~ s'appelle
autrement, renommez-le et modifiez le nom dans \verb~cos.sh~ (si vous êtes
sous Linux/MacOS) et \verb~cos.bat~ (si vous êtes sous Windows).
\subsubsection{Autres fichiers}
\label{sec-7-1-2}
Le nom des autres fichiers et leur localisation peut être changé en
modifiant le champ correspondant à ce fichier dans le fichier
\verb~configuration.txt~.

Imaginons, par exemple, que vous n'êtes pas satisfait du fait que le
fichier des retours des étudiants s'appelle \verb~reponsesEtudiants.txt~ et
est stocké dans \verb~Phase_2_entree~. Il faut la partie correspondante de
\verb~configuration.txt~. Elle vaut actuellement :
\begin{verbatim}
# Nom du fichier contenant la version remplie des fiches nominatives
# des etudiant·e·s (cf. outputListeFichesEtudiants.txt)
Phase_2_entree/reponsesEtudiants.txt
\end{verbatim}

Si vous voulez que le fichier s'appelle désormais
\verb~retoursDesEtudiants.txt~ et soit stocké au même niveau que le fichier
\verb~configuration.txt~, il faut modifier \verb~configuration.txt~ de la
manière suivante :
\begin{verbatim}
# Nom du fichier contenant la version remplie des fiches nominatives
# des etudiant·e·s (cf. outputListeFichesEtudiants.txt)
retoursDesEtudiants.txt
\end{verbatim}
\subsection{Changement des valeurs par défaut}
\label{sec-7-2}
Nous avons déjà évoqué comment (et pourquoi) changer dans
\verb~configuration.txt~:
\begin{itemize}
\item \verb~Bonus sur note etudiant·e si l'etudiant·e juge    critere comme l'encadrant~
\item \verb~Nombre de points quand encadrant·e estime que critere revele une bonne maitrise~
\item \verb~Nombre de points quand encadrant·e estime que critere requiert ameliorations mineures~
\item \verb~Nombre de points quand encadrant·e estime que critere requiert ameliorations majeures~
\end{itemize}

\verb~configuration.txt~ contient également la configuration des lignes
générées dans \verb~listeFichesEtudiants.txt~ lors de la phase 1 :
\begin{verbatim}
# Delimiteur entre les etudiants dans le fichier avec toutes les fiches nominatives
==================================================

# Delimiteur entre les soutenances dans les fichiers avec les fiches (nominatives ou generiques)
---------------------------------------------------------------------------------------

# Delimiteur des commentaires positifs dans les fichiers avec les fiches (nominatives ou generiques)
# NB : ce delimiteur ne doit pas commencer par le caractere '+' (plus) ou '-' (moins)
Commentaire/Justification du +

# Delimiteur des commentaires negatifs dans les fichiers avec les fiches (nominatives ou generiques)
# NB : ce delimiteur ne doit pas commencer par le caractere '+' (plus) ou '-' (moins)
Commentaire/Justification du -
\end{verbatim}
\section{Conclusion}
\label{sec-8}
Avec \emph{COS}, nous vous souhaitons des soutenances encore plus
intéressantes qu'avant ! N'hésitez pas à nous faire des retours sur
 \href{https://github.com/simatic/COS}{github} ou à \href{mailto:Michel.Simatic@telecom-sudparis.eu}{Michel.Simatic@telecom-sudparis.eu}.
% Emacs 24.5.1 (Org mode 8.2.10)
\end{document}
